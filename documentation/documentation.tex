\begin{page}
\begin{document}
\title{MarTeX}
	\hline
		MarTeX is a versitale TeX to HTML interpreter. 
		It comes with a few basic modules, but can easily be expanded upon. 
		It was written by Thijs Miedema for use with marie-curie.nl.
\section{Modules}
\subsection{Module Document}
The document module exposes the basic commands that are designed to mimick the LaTeX main environment.

\subsubsection{Implemented commands}
\begin{itemize}
    \item \descriptor{global}{\command{title}{Title}}{Make a big fancy title.}
    \item \descriptor{global}{\command{section}{Title}}{Make a section title.}
    \item \descriptor{global}{\command{subsection}{Title}}{Make a subsection title.}
    \item \descriptor{global}{\command{subsubsection}{Title}}{Make a subsubsection title.}
    \item \descriptor{global}{\command{textbf}{Text}}{Render Text in \textbf{bold}.}
    \item \descriptor{global}{\command{textit}{Text}}{Render Text in \textit{italics}.}
    \item \descriptor{global}{\command{underline}{Text}}{Render Text \underline{underlined}.}
    \item \descriptor{global}{\command{smallcaps}{Text}}{Render Text \smallcaps{smallcaps}.}
    \item \descriptor{global}{\command{hline}}{Make a horizontal line.}
    \item \descriptor{global}{\command{newline}}{Line break, equivalent to a double backslash.}
    \item \descriptor{global}{\command{paragraph}{text}}{Render the text as a paragraph, normally not noticable, but might have implications if a stylesheet is loaded. 
                                                        Use for a single line of text without commands only, please use the environment \textit{paragraph} otherwise.}
    \item \descriptor{global}{\command{ref}{label}}{Reference a label (from an image, for example). Works with both forward and backwards declaration (The label can be both above and below the ref command). Outputs a question mark when the label is undefined.}
    \item \descriptor{global}{\command{define}{Variable}{Value}}{Set the global Variable to Value. Often used to change module behaviour.}
    \item \descriptor{global}{\command{href}{URL}{Text}}{Make a hyperlink to URL with text Text}
    \item \descriptor{global}{\command{colo(u)r}{type}{definition}{text}}{Change the colour of a piece of text. There are a few possibilities: \colour{rgb}{255,0,100}{\command{colour}{rgb}{255,0,100}{text}}, \colour{rgba}{255,100,255,0.7}{\command{colour}{rgba}{100,100,255,0.7}}, \colour{hsl}{200,50,50}{\command{colour}{hsl}{200,50,50}{text}}, \colour{hsla}{200,50,50,0.7}{\command{colour}{hsla}{200,50,50,0.7}{text}} and \colour{hex}{#ff00ff}{\command{colour}{hex}{#ff00ff}{text}}. Please note that hsl and hsla are not supported by all browsers (especially email clients).}
    \item \descriptor{global}{\command{colo(u)r}{definition}{text}}{Same as above, but allow MarTeX to guess your colour type. Will always choose rgb over hsl.}
\end{itemize}

\begin{paragraph}
    The figure module exposes commands that are designed to mimick the LaTeX package graphicx, with a few exceptions. 
    Maybe the most notable exception is that most of these commands only work inside a figure environment, and will result in "command not found" errors if you use them elsewhere.
\end{paragraph}

\subsubsection{Implemented commands}
\begin{itemize}
    \item \descriptor{figure}{\command{includegraphics}{path/to/image}}{Use to display an image.}
    \item \descriptor{figure}{\command{caption}{caption text}}{Use to display text under your image.}
    \item \descriptor{figure}{\command{label}{label text}}{Use to give your image a label, 
                                            so you can use \textit{\command{ref}{label}} to reference it.}
    \item \descriptor{figure}{\command{width}{Width}}{ Set the width of your image. 
                    Width can have the following formats: 10 (this is a percentage), 10px, 10cm, 10em.}
    \item \descriptor{figure}{\command{height}{Height}}{ Set the height of your image. 
                    Height can have the following formats: 10 (this is a percentage), 10px, 10cm, 10em. }
    \item \descriptor{figure}{\command{alttext}{alt text}}{Set the alt text, also called hovertext, of your image.}
\end{itemize}

\subsubsection{Settings}
\begin{paragraph}
    Reference labels are used to point to your image from the text. They put the following text in your caption: Figure N, where N is replaced with the number of your image. However, you might want to use a different text. For this you can access the global variable 'figureheader'. For example, you could do: \command{define}{figureheader}{image}. Now, the caption wil read: Image N. You should put the \command{define} above all figure environments.
\end{paragraph}

\subsubsection{Example}
\begin{code}[latex]\command{define}{figureheader}{image}
\command{begin}{figure}
    \command{includegraphics}{test.png}
    \command{caption}{This is a test image}
    \command{label}{testimage}
    \command{width}{120px}
    \command{height}{30}
    \command{alttext}{A mouse!}
\command{end}{figure} 
\end{code}
\end{document}
\end{page}