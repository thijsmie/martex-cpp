\begin{page}

\link{martex.css}{stylesheet}{text/css}

\begin{document}
\title{MarTeX}
	\hline
		MarTeX is a versitale TeX to HTML interpreter. 
		It comes with a few basic modules, but can easily be expanded upon. 
		It was written by Thijs Miedema for use with marie-curie.nl.
\section{Modules}
\subsection{Module Document}
The document module exposes the basic commands that are designed to mimick the LaTeX main environment.

\subsubsection{Implemented commands}
\begin{itemize}
    \item \descriptor{global}{\command{title}{Title}}{Make a big fancy title.}
    \item \descriptor{global}{\command{section}{Title}}{Make a section title.}
    \item \descriptor{global}{\command{subsection}{Title}}{Make a subsection title.}
    \item \descriptor{global}{\command{subsubsection}{Title}}{Make a subsubsection title.}
    \item \descriptor{global}{\command{textbf}{Text}}{Render Text in \textbf{bold}.}
    \item \descriptor{global}{\command{textit}{Text}}{Render Text in \textit{italics}.}
    \item \descriptor{global}{\command{underline}{Text}}{Render Text \underline{underlined}.}
    \item \descriptor{global}{\command{smallcaps}{Text}}{Render Text \smallcaps{smallcaps}.}
    \item \descriptor{global}{\command{hline}}{Make a horizontal line.}
    \item \descriptor{global}{\command{newline}}{Line break, equivalent to a double backslash.}
    \item \descriptor{global}{\command{paragraph}{text}}{Render the text as a paragraph, normally not noticable, but might have implications if a stylesheet is loaded. 
                                                        Use for a single line of text without commands only, please use the environment \textit{paragraph} otherwise.}
    \item \descriptor{global}{\command{ref}{label}}{Reference a label (from an image, for example). Works with both forward and backwards declaration (The label can be both above and below the ref command). Outputs a question mark when the label is undefined.}
    \item \descriptor{global}{\command{define}{Variable}{Value}}{Set the global Variable to Value. Often used to change module behaviour.}
    \item \descriptor{global}{\command{href}{URL}{Text}}{Make a hyperlink to URL with text Text}
    \item \descriptor{global}{\command{colo(u)r}{type}{definition}{text}}{Change the colour of a piece of text. There are a few possibilities: \colour{rgb}{255,0,100}{\command{colour}{rgb}{255,0,100}{text}}, \colour{rgba}{255,100,255,0.7}{\command{colour}{rgba}{100,100,255,0.7}}, \colour{hsl}{200,50,50}{\command{colour}{hsl}{200,50,50}{text}}, \colour{hsla}{200,50,50,0.7}{\command{colour}{hsla}{200,50,50,0.7}{text}} and \colour{hex}{#ff00ff}{\command{colour}{hex}{#ff00ff}{text}}. Please note that hsl and hsla are not supported by all browsers (especially email clients).}
    \item \descriptor{global}{\command{colo(u)r}{definition}{text}}{Same as above, but allow MarTeX to guess your colour type. Will always choose rgb over hsl.}
\end{itemize}

\subsubsection{Special characters}
MarTeX has some special characters you need to escape to render:

\begin{tabular}[cc]
\begin{tabular}[|l | l |]
\hline
\textbf{Command}  &  \textbf{Result} \\
\hline \hline
\command{\&}     & \& \\
\command{\%}	 & \% \\
\command{\$}   	 & \$ \\
\command{copy}	 & \copy \\
\command{\>}	 & \> \\
\command{\<}	 & \< \\
\command{\}}	 & \} \\
\command{\{}	 & \{ \\
\command{cdot}	 & \cdot \\
\command{euro}	 & \euro \\
\command{pound}	 & \pound \\
\command{deg}	 & \deg \\
\command{backslash} & \backslash \\
\hline
\end{tabular}
&
\begin{tabular}[|l | l |]
\hline
\textbf{Command}  &  \textbf{Result} \\
\hline \hline
\command{"}AaEeIiOoUuy & \"A\"a\"E\"e\"I\"i\"O\"o\"U\"u\"y \\
\command{'}AaEeIiOoUuZ & \'A\'a\'E\'e\'I\'i\'O\'o\'U\'u\'Z \\
\command{`}AaEeIiOoUu & \`A\`a\`E\`e\`I\`i\`O\`o\`U\`u \\
\command{^}AaEeIiOoUu & \^A\^a\^E\^e\^I\^i\^O\^o\^U\^u \\
\command{~}AaIiOoNnUu & \~A\~a\~I\~i\~O\~o\~N\~n\~U\~u \\
\command{.}AaUu & \.A\.a\.U\.u \\
\command{-O}\command{-o} & \-O\-o \\
\command{ss} & \ss \\
\command{Alpha}\command{alpha} & \Alpha\alpha \\
\command{Beta}\command{beta} & \Beta\beta \\
... & ... \\
\command{Phi}\command{phi} & \Phi\phi \\
\command{Omega}\command{omega} & \Omega\omega \\
\hline
\end{tabular}
\\
\end{tabular}

\subsection{Module Figure}
	
\begin{paragraph}
    The figure module exposes commands that are designed to mimick the LaTeX package graphicx, with a few exceptions. 
    Maybe the most notable exception is that most of these commands only work inside a figure environment, and will result in "command not found" errors if you use them elsewhere.
\end{paragraph}

\subsubsection{Implemented commands}
\begin{itemize}
    \item \descriptor{figure}{\command{includegraphics}{path/to/image}}{Use to display an image.}
    \item \descriptor{figure}{\command{caption}{caption text}}{Use to display text under your image.}
    \item \descriptor{figure}{\command{label}{label text}}{Use to give your image a label, 
                                            so you can use \textit{\command{ref}{label}} to reference it.}
    \item \descriptor{figure}{\command{width}{Width}}{ Set the width of your image. 
                    Width can have the following formats: 10 (this is a percentage), 10px, 10cm, 10em.}
    \item \descriptor{figure}{\command{height}{Height}}{ Set the height of your image. 
                    Height can have the following formats: 10 (this is a percentage), 10px, 10cm, 10em. }
    \item \descriptor{figure}{\command{alttext}{alt text}}{Set the alt text, also called hovertext, of your image.}
\end{itemize}

\subsubsection{Settings}
\begin{paragraph}
    Reference labels are used to point to your image from the text. They put the following text in your caption: Figure N, where N is replaced with the number of your image. However, you might want to use a different text. For this you can access the global variable 'figureheader'. For example, you could do: \command{define}{figureheader}{image}. Now, the caption wil read: Image N. You should put the \command{define} above all figure environments.
\end{paragraph}

\subsubsection{Example}
\begin{code}[latex]\command{define}{figureheader}{image}
\command{begin}{figure}
    \command{includegraphics}{test.png}
    \command{caption}{This is a test image}
    \command{label}{testimage}
    \command{width}{120px}
    \command{height}{30}
    \command{alttext}{A mouse!}
\command{end}{figure} 
\end{code}

\subsection{Module Itemize}

\begin{paragraph}
    The itemize module exposes commands to help make fancy lists.
\end{paragraph}

\subsubsection{Implemented commands}
\begin{itemize}
    \item \descriptor{itemize/enumerate}{\command{item}{(optional)Itemtext}}{Add an item to a list.}
    \item \descriptor{itemize/enumerate}{\command{setmarker}{type}}{Set the marker type. Valid types are: square, bullet, circle, none, numbers, letters, LETTERS, roman, ROMAN.}
\end{itemize}

\subsubsection{Implemented environments}
\begin{paragraph}
The itemize module implements two environments. They are equivalent, but have a different default marker type.
\end{paragraph}
\begin{itemize}
    \item \envdescriptor{itemize}{A list with default marker 'bullet'}
    \item \envdescriptor{paragraph}{A list with default marker 'numbers'}
\end{itemize}

\subsubsection{Example}

\begin{tabular}[l|l]
Code & Rendered \\
\hline
\begin{code}[latex]
\command{begin}{itemize}
    \command{item}{Akcie}
    \command{item}{Ambicie}
    \command{item}{Batavieren}
    \command{item}{Boson}
\command{end}{itemize}
\command{begin}{itemize}
    \command{setmarker}{square}
    \command{item} Brouwcommissie
    \command{item} Colloquiumcommissie
    \command{item} Excurcie
\command{end}{itemize}
\command{begin}{enumerate}
    \command{item}{Impuls}
    \command{item}{Lustrumcie}
    \command{item}{M\"unchencie}
\command{end}{enumerate}
\command{begin}{enumerate}
    \command{setmarker}{letters}
    \command{item}{Ouderdagcie}
    \command{item}{Particie}
    \command{item}{PR-cie}
    \command{item}{Reunistencie}
\command{end}{enumerate}
\command{begin}{enumerate}
    \command{setmarker}{ROMAN}
    \command{item} Sponsorcommissie
    \command{item} Symposiumcie
    \command{item} Vakancie
    \command{item} Weekendcommissie
    \command{item} WWW-cie
    \command{item} Co\"ordinatiecommissie
\command{end}{enumerate}
\end{code}
&
\begin{itemize}
    \item{Akcie}
    \item{Ambicie}
    \item{Batavieren}
    \item{Boson}
\end{itemize}
\begin{itemize}
    \setmarker{square}
    \item{Brouwcommissie}
    \item{Colloquiumcommissie}
    \item{Excurcie}
\end{itemize}
\begin{enumerate}
    \item{Impuls}
    \item{Lustrumcie}
    \item{M\"unchencie}
\end{enumerate}
\begin{enumerate}
    \setmarker{letters}
    \item{Ouderdagcie}
    \item{Particie}
    \item{PR-cie}
    \item{Reunistencie}
\end{enumerate}
\begin{enumerate}
    \setmarker{ROMAN}
    \item{Sponsorcommissie}
    \item{Symposiumcie}
    \item{Vakancie}
    \item{Weekendcommissie}
    \item{WWW-cie}
    \item{Co\"ordinatiecommissie}
\end{enumerate}
\\
\end{tabular}

\subsection{Module Tabular}

\end{document}
\end{page}