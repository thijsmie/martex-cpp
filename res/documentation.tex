\begin{page}
%
\link{martex-page.css}{stylesheet}{text/css}
\link{martex.css}{stylesheet}{text/css}
\meta{viewport}{width=device-width, initial-scale=1.0}
%
\newcommand\descriptor[2]{
\textbf{#1}
\begin{paragraph}#2\end{paragraph}%
}%
%
\begin{document}
\title{MarTeX-Cpp}

MarTeX-Cpp is a flexable, extendable and fast \textit{TeX-to-HTML} converter by fully tokenizing-lexing-parsing-interpreting the TeX source. It can be compiled to serve as a module for different languages, like PHP or JavaScript. It's goal is to provide a way to allow users to write TeX they are familiar with to produce nice looking pages like this one, but without allowing code injection.
        
\section{Styling}

\begin{itemize}
    \item Headings
    \begin{itemize}
    \item \descriptor{\command{title}{Title}}{Make a big fancy title.}
    \item \descriptor{\command{chapter}{Title}}{Make a chapter title.}
    \item \descriptor{\command{section}{Title}}{Make a section title.}
    \item \descriptor{\command{subsection}{Title}}{Make a subsection title.}
    \item \descriptor{\command{subsubsection}{Title}}{Make a subsubsection title.}
    \end{itemize}
    \item Markup
    \begin{itemize}
    \item \descriptor{\command{textbf}{Text} or \command{bold}{Text}}{Render Text in \textbf{bold}.}
    \item \descriptor{\command{textit}{Text} or \command{italic}{Text}}{Render Text in \textit{italics}.}
    \item \descriptor{\command{textul}{Text} or \command{underline}{Text}}{Render Text \underline{underlined}.}
    \item \descriptor{\command{textsc}{Text} or \command{smallcaps}{Text}}{Render Text \smallcaps{smallcaps}.}
    \item \descriptor{\command{textms}{Text} or \command{mono}{Text}}{Render Text \mono{monospaced}.}
    \end{itemize}
    \item Elements
    \begin{itemize}
    \item \descriptor{\command{hline}}{Make a horizontal line.}
    \item \descriptor{\command{newline}}{Line break, equivalent to a double backslash.}
    \item \descriptor{\command{ref}{label}}{Reference a label (from an image, for example). Works with both forward and backwards declaration (The label can be both above and below the ref command). Outputs a question mark when the label is undefined.}
    \item \descriptor{\command{define}{Variable}{Value}}{Set the global Variable to Value. Often used to change module behaviour.}
    \item \descriptor{\command{href}{URL}{Text}}{Make a hyperlink to URL with text Text}
    \item \descriptor{\command{colo(u)r}{type}{definition}{text}}{Change the colour of a piece of text. \colour{ff00ff}{\command{colour}{ff00ff}{text}}}
    \end{itemize}
\end{itemize}

\section{Special Characters}

\begin{tabular}[|l | l |c| l |l |] \width{100}
\hline
\textbf{Command}  &  \textbf{Result} && \textbf{Command} & \textbf{Result} \\
\hline \hline
\command{\&}        & \&        && \command{"}AaEeIiOoUuy   & \"A\"a\"E\"e\"I\"i\"O\"o\"U\"u\"y \\
\command{\%}        & \%        && \command{'}AaEeIiOoUuZ   & \'A\'a\'E\'e\'I\'i\'O\'o\'U\'u\'Z \\
\command{\$}        & \$        && \command{`}AaEeIiOoUu    & \`A\`a\`E\`e\`I\`i\`O\`o\`U\`u    \\
\command{copy}      & \copy     && \command{^}AaEeIiOoUu    & \^A\^a\^E\^e\^I\^i\^O\^o\^U\^u    \\
\command{\>}        & \>        && \command{^}AaEeIiOoUu    & \^A\^a\^E\^e\^I\^i\^O\^o\^U\^u    \\
\command{\<}        & \<        && \command{~}AaIiOoNnUu    & \~A\~a\~I\~i\~O\~o\~N\~n\~U\~u    \\
\command{\}}        & \}        && \command{.}AaUu          & \.A\.a\.U\.u                      \\
\command{\{}        & \{        && \command{-O}o            & \-O\-o                            \\
\command{cdot}      & \cdot     && \command{ss}             & \ss                               \\
\command{euro}      & \euro     && \command{Alpha}\command{alpha}   & \Alpha\alpha              \\
\command{pound}     & \pound    && \command{Beta}\command{beta}     & \Beta\beta                \\
\command{deg}	    & \deg      &&  ...                             &                           \\
\command{backslash} & \backslash&& \command{Omega}\command{omega}   & \Omega\omega              \\
\hline
\end{tabular}

\section{Images}
\begin{paragraph}
    Images are pretty. Normally the images try to fill the page width, you can control this behaviour with the provided
    sizing options. If you add a caption to an image it will put "Figure 1: caption" under it. You can change "Figure" to
    , for example, "Picture" by setting the header with a \command{define}{header}{Picture }.
\end{paragraph}
\begin{itemize}
    \item \descriptor{\command{includegraphics}{url}}{Use to display an image.}
    \item \descriptor{\command{caption}{caption text}}{Use to display text under your image.}
    \item \descriptor{\command{label}{label text}}{Use to give your image a label, 
                                            so you can use \command{ref}{label} to reference it.}
    \item \descriptor{\command{width}{Width}}{ Set the width of your image. 
                    Width can have the following formats: 10 (this is a percentage), 10px, 10cm, 10em.}
    \item \descriptor{\command{height}{Height}}{ Set the height of your image. 
                    Height can have the following formats: 10 (this is a percentage), 10px, 10cm, 10em. }
    \item \descriptor{\command{alttext}{alt text}}{Set the alt text, also called hovertext, of your image.}
\end{itemize}

\begin{code}[latex]
\command{begin}{figure}
    \command{includegraphics}{test.png}
    \command{caption}{This is a test image}
    \command{label}{testimage}
    \command{width}{120px}
    \command{height}{30}
    \command{alttext}{A mouse!}
\command{end}{figure} 
\end{code}

\section{Lists}
\begin{paragraph}
    Note that there are two different list environments: \mono{enumerate} and \mono{itemize} with a different default markers, respectively numbers and bullets.
\end{paragraph}
\begin{itemize}
    \item \descriptor{\command{item}}{Mark the start of the next item in the list.}
    \item \descriptor{\command{setmarker}{type}}{Set the marker type. Valid types are: square, bullet, circle, none, numbers, letters, LETTERS, roman, ROMAN.}
\end{itemize}

\begin{code}[latex]
\command{begin}{itemize}
    \command{setmarker}{square}
    \command{item} Apple
    \command{item} Banana
    \command{item} Kiwi 
    \command{item} Pear
\command{end}{itemize}
\end{code}

\section{Tables}

\begin{paragraph}
    Tables are a good way to represent content in a neat way. In MarTeX they behave pretty similar to the LaTeX tabulars. The tabular format accepts '|' for vertical lines, 'l' for left aligned column, 'c' for a centered column and 'r' for a right aligned column.
\end{paragraph}

\begin{itemize}
    \item \descriptor{\command{hline}}{Add a horizontal line at this position in the table. Maximum of two lines per row.}
    \item \descriptor{\command{multicolumn}{N}{format}{text}}{Make a cell span over N columns. You can add special formatting to this cell (so just like your table formatting).}
    \item \descriptor{\command{multirow}{N}{text}}{Make a cell span over N rows}
\end{itemize}

\begin{code}[latex]
\command{begin}{tabular}\[|| l | c | r ||\]
    \command{hline} \command{hline}
    1 & 2 & 3 \backslash\backslash \command{hline}
    4 & 5 & 6 \backslash\backslash \command{hline}
    7 & 8 & 9 \backslash\backslash \command{hline}
    \command{multicolumn}{3}{| c |}{0} \backslash\backslash
    \command{hline}
\command{end}{tabular}
\end{code}

\section{Contributing}

Please feel free to submit anything! The only thing I ask is that you include at least one test that verifies the behaviour of your change.

\subsection{Bug Bounty Program}

If you find a way insert arbitrary html tags and/or get javascript execution from a certain input string then first I'd like to congratualate you on a job well done! Please contact me privately with said input and we'll work on pushing out a fix as quickly as possible. Because I'm still a student there is unfortunately no hard cash to be made here, but certainly there is coffee and cake in it for you!

\section{links}

\begin{itemize}
   \item \href{http://tmiedema.com/martex}{Live MarTeX interpreter}
   \item \href{http://tmiedema.com/martex/development.html}{Development Guide}
\end{itemize}

\section{Authors}

\begin{itemize}
  \item \bold{Thijs Miedema} - \italic{Initial work} - \href{https://github.com/thijsmie}{thijsmie}
\end{itemize}

\section{Acknowledgments}

Thanks to my committee for their endless patience and enthusiasm while listening to me whining on about this.
\\
Thanks to \href{http://www.iport.it}{\textit{Diego Perini}} for writing a nifty regex-based URL validator.

\section{License}

This project is licensed under the MIT License - see the \href{LICENSE.md}{LICENSE.md} file for details


\end{document}
\end{page}