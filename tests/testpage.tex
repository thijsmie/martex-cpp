\section{E-mail}
\subsection{Twee e-mailadressen?}

Als het goed is, heb je van de universiteit twee e-mailadressen gekregen. De \`e\`en is van de centrale studentenadministratie en heeft een vorm als voornaam.achternaam@student.ru.nl. Het andere heb je van de faculteit NWI gekregen, en is iets als voorletterachternaam@science.ru.nl. Als dit adres al door iemand anders bezet is, kan het zijn dat je iets anders hebt gekregen.
\\
\\
Beide accounts zijn goed bruikbaar. Ze zijn beide via webmail bereikbaar, zodat je zelfs op vakantie je e-mail kunt lezen.
\\
\\
Voor de centrale Radboud-mail ga je naar "mail.ru.nl": \href{www.mail.ru.nl}{www.mail.ru.nl}. Voor je science mail kun je twee applicaties gebruiken: "squirrel":\href{www.squirrel.science.ru.nl}{www.squirrel.science.ru.nl} is een simpele interface, en "roundcube":\href{www.roundcube.science.ru.nl}{www.roundcube.science.ru.nl} kan wat meer.
\\
\\
Ze hebben als voordeel dat ze een grote opslagcapaciteit hebben. Daarnaast kun je je eigen e-mailprogramma gebruiken om e-mail naar je eigen pc te downloaden (zowel voor mail.ru.nl als voor science.ru.nl).

\\
\\
\subsection{E-mail bekijken op de faculteit}

Als je fan bent van de linux commandline, kun je je mail lezen met "Alpine". Je start dit programma met het commando \textbf{alpine}. Je ziet nu een menu. Als je je mail wilt lezen, wat meestal het geval is, kies je nu voor "Message index" (\textbf{I}). Je krijgt nu alle mailtjes te zien die de afgelopen tijd naar je zijn verstuurd. Met de pijltjestoetsen en de enter-toets kun je ze lezen.
\\
\\

Merk op dat onderin het venster altijd een lijst met mogelijke commando's staat! Op die manier raak je nooit echt de weg kwijt.
\\
\\
Post beantwoorden doe je met de \textbf{R}-toets, een nieuw bericht verstuur je met \textbf{C}. Een handige optie van dit programma is het adresboek. Druk in het hoofdmenu op \textbf{A} (of ga met de pijltjestoetsen naar het menu-item en druk op enter) en vervolgens op @ om een nieuwe entry te maken. Vul onder "nickname" een korte naam in en de rest van de gegevens, en je hoeft als je iemand post wil sturen, alleen diens nickname in te typen als geadresseerde.
\\
\\

Een ander mailprogramma dat je kunt gebruiken, is \textbf{thunderbird}. Deze is een stuk langzamer dan alpine, maar lijkt wel meer op de Windows-programma's die veel mensen gewend zijn. 
\\
\\
\subsection{Forwarden van e-mail}

Welk adres kun je nu het beste gebruiken? Als je veel gebruik denkt te gaan maken van de computerfaciliteiten op de faculteit, kun je het beste je science-adres actief gebruiken en je student-post forwarden naar dit adres. Andersom kan natuurlijk ook. Op die manier hoef je maar \`e\`en van beide adressen te checken. 
\\
\\
\subsubsection{E-mail forwarden van je student.ru.nl-account}

Ga naar share.ru.nl en kies "Voorkeuren". Dan druk links op de knop Mail (het lijkt alleen een uitklap knop, maar er zitten ook instellingen achter). Onder "Ontvangen berichten" staat een balkje "Een kopie doorsturen naar:" Hier kun je je science-adres invullen - of een willekeurig ander adres uiteraard. 

\\
\\
\subsubsection{E-mail forwarden van je science.ru.nl-account}

Dit kun je eenvoudig zelf instellen op \href{www.dhz.science.ru.nl}{www.dhz.science.ru.nl}. Log daar in met je faculteitsinloggegevens.

